\documentclass[10pt,letterpaper]{article}
\usepackage[utf8]{inputenc}
\usepackage[T1]{fontenc}
\usepackage{amsmath}
\usepackage{amssymb}
\usepackage{graphicx}
\usepackage{tikz}
\usetikzlibrary{shapes, arrows, positioning}
\title{Flowcharts}
\author{Jarno van der Kolk}
\begin{document}
	\maketitle
	\section{Introduction}
	This demonstrates how to create flow charts. It requires the \texttt{tikz} package.
	\begin{figure}
		% Define a node
		\tikzstyle{block} = [rectangle, draw, fill=gray!5, 
		text width=.4\columnwidth, text centered, rounded corners, minimum height=2em]
		\tikzstyle{line} = [thick,draw, -latex']
		\begin{tikzpicture}[node distance = 3em and 4em, auto]
			% Draw the blocks. Note that the "block" shape is defined earlier
			\node[block] (init) {Start};
			\node[block, below = of init] (step1) {Step 1};
			\node[block, below = of step1] (step2) {Step 2};
			\node[block, right = of step2] (step2a) {Step 2A};
			\node[block, below = of step2] (step3) {Step 3};
			\node[block, below = of step3] (end) {End};

			% Draw the lines connecting the block. Again, "line" is defined earlier
			\path[line] (init) to (step1);
			\path[line] (step1) to (step2);
			\path[line] (step2) to node {No} (step3);
			\path[line] (step2) to node {Yes} (step2a);
			\path[line] (step2a) to[out=270,in=0] (step3);
			\path[line] (step3) to (end);
		\end{tikzpicture}
		\caption{Example of a flow chart}
	\end{figure}
\end{document}
